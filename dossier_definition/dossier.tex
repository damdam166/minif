\documentclass[48pt]{article}
\usepackage[frenchb]{babel}
\usepackage{graphicx}
\usepackage{subcaption}
\usepackage{amssymb}

\date{}

\begin{document}

\title{\Huge Minifusée : Ail et fines'eirb}
\maketitle

% To display the number of the page
\pagestyle{plain}

\vspace{4\baselineskip}

% To display the members of the group
\begin{center}
	\Large DELPY Damien

	\Large DERAMAT Louis

	\Large YE William

	\Large HE Jacky

	\Large HARDOIN Ayman

	\Large KERACH Shakty
\end{center}

% The Eirspace logo
\begin{figure}
	\begin{center}
		\includegraphics[width=0.4\textwidth]{pics/eirspace.png}	
	\end{center}
\end{figure}


\newpage

\section{\centering \Large Présentation d'Ail et fines'eirb}

\vspace{2\baselineskip}

Nous sommes des étudiants de l'école d'ingénieur de Bordeaux nommée : \centering ENSEIRB MATMECA.

\vspace{2\baselineskip}

Plusieurs d'entre nous essayent actuellement de passer le certificat Espace, et ce projet est nécessaire à l'obtention de ce dernier.

Cela permettra aussi aux autres membres d'appliquer des concepts vu en cours, et d'apprendre à réaliser des projets en équipe.

\vspace{2\baselineskip}

Nous venons d'une multitude de filière : 

Informatique, Électronique, Mathématiques et Mécaniques, et Télécommunications.

\vspace{2\baselineskip}

\section{\Large Objectifs}

\vspace{2\baselineskip}

Nous voulons : 

\begin{enumerate}
	\item Dans le cadre Mathématiques et Mécaniques : Trouver le point de culmination de la fusée à l'aide d'un altimètre (et d'une horloge) en temps réel, puis comparer avec les valeurs déterminées par StabTraj.

	\item Dans le cadre Électronique : installer une antenne sur notre fusée pour emettre des ondes wifi sur une grande distance.

	\item Dans le cadre Mathématiques et Mécaniques : évaluer la température maximale dans la fusée, et comparer la valeur avec celle trouvée par la théorie.

	\item Dans le cadre Télécommunications : Diffuser en direct un live de bonne qualité sur Youtube : \centering une caméra et tous les capteurs en temps réels.
\end{enumerate}
	
\newpage

\section{\centering \Large Spécifications Principales}

\vspace{2\baselineskip}

\begin{figure}[h]
		\begin{center}
		\begin{tabular}{|c|c|c|c|}
	  	\hline
			& Dimensions & Matériaux & Spécifications \\
 	 	\hline
			Peau & Diamètre = 80mm, Hauteur = 900mm & PVC & tube de 1mm d'épaisseur \\
		  \hline
			Propulseur & inconnu & non spécifié & Pandora Pro 24G \\
	 	 \hline
			Coiffe & D = 70mm , H = 100mm & PLA & Ogive imprimée en 3D \\
		  \hline
			Bagues & Diamètre = 78 mm & Bois & 10mm d'épaisseur \\
 		 \hline
			Ailerons & Emplanture = 100mm & Bois & 4 ailerons, 4mm d'épaisseur \\
 				
			& Saumon = 70 mm & & \\
			& Flèche = 110 mm & & \\
			& Envergure = 140 mm & & \\
			
		\hline
		\end{tabular}
	\end{center}

	\caption{tableau des spécifications}
\end{figure}

\vspace{2\baselineskip}

\section{\centering \Large Appareils supplémentaires}

Nous aurons besoin d'une carte réseau pour la transmission : 

\centering Module LoRa mkr 32.

\end{document}
